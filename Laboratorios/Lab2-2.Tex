\documentclass{article}

\usepackage{graphicx}
\usepackage{tikz}
\usepackage{pgfplots}
\usepackage{geometry}
\usepackage{fancyhdr}
\usepackage{graphicx}
\usepackage[hidelinks]{hyperref}
\usepackage{setspace}
\usepackage{mathtools}
\usepackage{amsfonts} % para símbolo de los naturales
\usepackage{SIunitx}
\usepackage{enumitem}
\usepackage[spanish, es-noshorthands, es-noquoting]{babel}
\usepackage{xcolor}
\pagestyle{fancy}
\fancyhf{}
\setlength{\headheight}{70.38103pt}
\rhead{\textit{David G., Laura R., Luisa R., María V.}}
\lhead{\includegraphics[width = 4cm]{\logo}}
\lfoot{Página \thepage}
\rfoot{Calorimetría}
\renewcommand{\headrule}{\hbox to \headwidth{\color{rojoEci}\leaders\hrule height \headrulewidth\hfill}}
\renewcommand{\footrulewidth}{0.4pt}

\hyphenpenalty=10000

\newcommand{\logo}{"C:/Users/lalal/OneDrive/Documentos/Universidad/logo-eci.png"}

\newcommand{\titlename}{Calorimetría}%
\renewcommand{\author}{{David Gómez, Laura Rincón, Luisa Rodríguez, María Vivas}}

\definecolor{rojoEci}{RGB}{225, 70, 49}

% Temporal func

\NewDocumentCommand{\partiald}{ o m m }%
    {
        \IfNoValueTF{#1}
        {
            \dfrac{{\partial} #2}{{\partial} #3}
        }{
            \dfrac{{\partial}^{#1} #2}{\partial {#3}^{#1}}
        }        
    }

% Enumi func

\renewcommand{\labelenumi}{(\Roman{enumi})}
\renewcommand{\labelenumii}{(\roman{enumii})}

\doublespacing
\begin{document}
\begin{titlepage}
    \begin{center}
        \vspace{1cm}

        \textbf{\Huge{\titlename}}

        \vspace{1.5cm}

        \textbf{\large{\author}}

        \vspace{3cm}

        \includegraphics[width=0.8\textwidth]{\logo}
        
        \vfill

        Física de Calor y Ondas

        Escuela Colombiana de Ingeniería Julio Garavito 

        \today
    \end{center}
\end{titlepage}

\section{Experimento}

El experimento consiste en calcular la capacidad calórica que tiene
un cubo de un material específico a partir de la temperatura del Estado
de equilibrio que se alcanza cuando se mete el cubo a $\SI{50}{\degreeCelsius}$
en un beacker con agua a temperatura ambiente.

\section{Teoría}

Idealmente, la energía tipo calor que tiene el cubo
$Q_c$ dependería únicamente de la energía tipo calor que le transmite
al agua y al beacker, es decir, no hay fugas de energía hacia el entorno
por lo que idealmente podemos definir lo siguiente:

\begin{align*}
    & Q_c = Q_H + Q_b\\
    \equiv& \left<Q_c > Q_h \text{, entonces es negativo porque transfiere} \right> \\
    & - m_c c_c (T_f - T_{cf}) = m_h c_h(T_f - T_a) + m_b c_b (T_f-T_a)\\
    \equiv& \left< Despejando c_c \right>\\
    & c_c = -\frac{(m_Hc_H + m_bc_b)(T_f-T_a)}{m_c(T_{cf}-T_{f})}\\
    \equiv&\\
    & c_c = \frac{(m_hc_h + m_bc_b)(T_f-T_a)}{m_c(T_f-T_{cf})}
\end{align*}

Sin embargo, a nivel experimental eso no se puede lograr, es decir,
no se puede despreciar tan facilmente la transferencia de enrgía 
tipo calor al entorno, por lo que toca hacer una serie de cálculos
extra. Como indica la imagen, en condiciones ideales 
se debería llegar a que después de un cierto tiempo $t_1$ el sistema 
cubo con el sistema agua + beacker entran en equilibrio a una temperatura
$T_f$. Pero al haber transferencia de calor hacia el entorno, primero
no se puede llegar al valor $T_f$ y segundo, el sistema no entrará en equilibrio
a menos que llegue a la temperatura ambiente nuevamente. Aún así, podemos 
estimar cuál es el $T_f$ ideal calculando integrales y relaciones.


\includegraphics[width=1\textwidth]{C:/Users/lalal/OneDrive/Documentos/Universidad/FCOP/Laboratorios/GraficaTemperaturas.png}


Tenemos, por la Ley de Enfriamiento de Newton que:

\begin{align*}
    & \frac{dQ_E}{dt} = K(T_A(t) - T_a)\\
    \Rightarrow& \left<\text{Integrando} \right> \\
    & Q_E = \int_{0}^{Qt} dQ_E = \int_{0}^{t} K(T_A(t) - T_a) dt\\
    \equiv& \\
    & Q_{E_1} = K\int_{0}^{t_1} (T_A(t) - T_a)dt = KA_1\\
    \land& \left< \text{De forma análoga} \right>\\
    & Q_{E_2} = K\int_{t_1}^{t_2} (T'_A (t) - T_A)dt = KA_2
\end{align*}

$A_1$ y $A_2$ los podemos calcular, ya que experimentalmente se puede 
aproximar el comportamiento de la gráfica. $\Delta T_2$ lo podemos calcular
ya que sería medir la temperatura en $t_1$, que sería la temperatura máxima que 
alcanza el sistema experimentalmente; luego medir la temperatura en el tiempo $t_2$ y 
restarlas.
Además tenemos que $\Delta T_1$ es a $\Delta T_2$ como $A_1$ es a $A_2$, de esta manera
podemos hayar $\Delta T_1$:

\begin{align*}
    & \frac{\Delta T_1}{\Delta T_2} = \frac{A_1}{A_2}\\
    \equiv& \\
    & \Delta T_1 = T_2 \frac{A_1}{A_2}
\end{align*}

Ya podemos calcular la temperatura de equilibrio que se alcanza
idealmente en el sistema. Como se ve en la imagen:

\[
    T_f = T_1 + T_{fR} = T_2 \frac{A_1}{A_2} + T(t_1)
\]
Así, ya podemos encontrar las constante calórica del cubo $c_c$.





\end{document}
