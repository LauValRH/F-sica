\documentclass{article}

\usepackage{geometry}
\usepackage{fancyhdr}
\usepackage{graphicx}
\usepackage[hidelinks]{hyperref}
\usepackage{setspace}
\usepackage{mathtools}
\usepackage{amsfonts} % para símbolo de los naturales
\usepackage{SIunitx}
\usepackage{enumitem}
\usepackage[spanish, es-noshorthands, es-noquoting]{babel}
\usepackage{xcolor}
\pagestyle{fancy}
\fancyhf{}
\setlength{\headheight}{70.38103pt}
\rhead{\textit{David G., Laura R., Luisa R., María V.}}
\lhead{\includegraphics[width = 4cm]{\logo}}
\lfoot{Página \thepage}
\rfoot{Resonancia en un resorte}
\renewcommand{\headrule}{\hbox to \headwidth{\color{rojoEci}\leaders\hrule height \headrulewidth\hfill}}
\renewcommand{\footrulewidth}{0.4pt}

\hyphenpenalty=10000

\newcommand{\logo}{"C:/Users/lalal/OneDrive/Documentos/Universidad/logo-eci.png"}

\newcommand{\titlename}{Resonancia \\[10pt] En Un Resorte}%
\renewcommand{\author}{{David Gómez, Laura Rincón, Luisa Rodríguez, María Vivas}}

\definecolor{rojoEci}{RGB}{225, 70, 49}

% Temporal func

\NewDocumentCommand{\partiald}{ o m m }%
    {
        \IfNoValueTF{#1}
        {
            \dfrac{{\partial} #2}{{\partial} #3}
        }{
            \dfrac{{\partial}^{#1} #2}{\partial {#3}^{#1}}
        }        
    }

% Enumi func

\renewcommand{\labelenumi}{(\Roman{enumi})}
\renewcommand{\labelenumii}{(\roman{enumii})}

\doublespacing
\begin{document}
\begin{titlepage}
    \begin{center}
        \vspace{1cm}

        \textbf{\Huge{\titlename}}

        \vspace{1.5cm}

        \textbf{\large{\author}}

        \vspace{3cm}

        \includegraphics[width=0.8\textwidth]{\logo}
        
        \vfill

        Física de Calor y Ondas

        Escuela Colombiana de Ingeniería Julio Garavito 

        \today
    \end{center}
\end{titlepage}

\clearpage
\tableofcontents
\clearpage

\section{Fundamentación}

La resonancia mecánica es un fenómeno que ocurre cuando 
se le aplica cierta fuerza armónica a un oscilador armónico 
con una frecuencia angular $w_n$ específica,
que hace que el sistema mecánico amortiguado alcance su pico
de amplitud. 

Las fuerzas que acúan en el sistema son una fuerza armónica 
$\vec{F}_A$ y una fuerza de fricción $\vec{F}_f$, donde
$\vec{F}_a = \vec{F}_Asen(wt)$ y 
$\vec{F}_f = -b\vec{v}$, siendo b el coeficiente de 
proporcionalidad entre fricción y velocidad para $v$ pequeñas.

La amplitud en el sistema está dada por: 
\[A(w) = \frac{F_A/m}{\sqrt{(w^2 - \frac{k}{m})^2 + (\frac{b}{m}w)^2 }}\]
y la amplitud máxima es en $A(w_{res})$:
\[A(w_{res}) = \frac{F_A/m}{\sqrt{((w_{res})^2 - \frac{k}{m})^2} + (\frac{b}{m}w_{res})^2}\]
donde $w_{res} = \sqrt{\dfrac{k}{m}}$ y $k = \dfrac{mg}{\Delta l_0}$

\clearpage

\section{Descripción}

En este laboratorio, el montaje se hizo de la siguiente manera:

Un resorte con una masa colgando de una polea. El otro 
extremo de la polea está conectada a un motor impulsado por un
circuito DC. Este motor es el que aplica la fuerza armónica.
Con un potenciómetro se mide el voltaje que se le aplica al
motor. Se toma el voltaje proporcional a la frecuencia angular.


\section{Mediciones}

\includegraphics[width=0.3\textwidth]{C:/Users/lalal/OneDrive/Documentos/Universidad/FCOP/Laboratorios/Sin título.png}

Se tiene que $W_{res}$ teórico es \[W_{res} = \sqrt{\frac{k}{m}}\]
que reemplazando con los datos dados 
\[W_{res} = \sqrt{\frac{0,388}{0,01387}} = 0,1673\]

y $w_{res}$ experimental es 

\[w_{res(exp)} = \frac{2\pi}{T}\]

que reemplazando con $T = 0,44$, nos queda que 
\[w_{res(exp)} = \frac{2\pi}{0,44} = 14,27\]


\section{Resultados}

Se logró observar el efecto de resonancia en algún
voltaje. Sin embargo, el resorte se movía tan 
violentamente que era muy complicado reconocer 
la medida de la amplitud y el tiempo de oscilación. Por este motivo, se puede 
decir que experimentalmente, al ser las medidas a ojo, 
tienen un error muy grande, y así es como el voltaje 
y la frecuencia halladas teóricamente distan tanto del
resutado experimental. 
 \end{document}